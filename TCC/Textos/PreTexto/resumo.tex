%%%% RESUMO
%%
%% Apresentação concisa dos pontos relevantes de um texto, fornecendo uma visão rápida e clara do conteúdo e das conclusões do
%% trabalho.

\begin{resumoutfpr}%% Ambiente resumoutfpr

Este trabalho apresenta o desenvolvimento e avaliação de modelos de aprendizado de máquina aplicados à classificação de objetos em linhas de transmissão, utilizando dados de profundidade capturados por uma câmera \textit{RealSense D415}, um sensor LiDAR \textit{RPLIDAR A1} da Slamtec e a combinação de ambos. Foram testados cinco modelos - \textit{k-Nearest Neighbors}, Árvore de Decisões, \textit{Naive Bayes}, Rede Neural e Floresta Aleatória - com dados simulados e reais. Os resultados indicaram que o uso de apenas um dos sensores já permite uma classificação satisfatória, mas a fusão sensorial torna o sistema mais robusto. Os modelos mais leves apresentaram desempenho competitivo, evidenciando seu potencial para futura implementação embarcada. O estudo também explora técnicas de pré-processamento de imagens e dados para a criação de \textit{features} utilizadas no treinamento dos modelos. Destaca-se, assim, a viabilidade do uso de sensores de profundidade na inspeção autônoma de linhas de transmissão, com ganhos em segurança e redução de custos operacionais.

\end{resumoutfpr}
%De acordo com a NBR 6028:2021, a apresentação gráfica deve seguir o padrão do documento no qual o resumo está inserido. Para definição das palavras-chave (e suas correspondentes em inglês no abstract) consultar em Termo tópico do Catálogo de Autoridades da Biblioteca Nacional, disponível em: http://acervo.bn.gov.br/sophia_web/autoridade